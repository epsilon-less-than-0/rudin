\documentclass[../../templates/chapter]{subfiles}

\begin{document}

\chapter*{Preface}\label{chap:00-preface}
\addcontentsline{toc}{chapter}{Preface}

This project is founded in the idea that mathematics, and resources for
learning mathematics, should be available to all. While Walter Rudin's
\emph{Principles of Mathematical Analysis}\cite{rudin} is a beautiful book,
there are many points in the book that may be unclear, especially to
individuals first learning analysis. It if of course, \emph{extremely}
important when learning mathematics, or any subject for that matter, to
struggle through material. That said, the point at which struggle is helpful,
is different for each individual.  We find the level of struggle expected by
Rudin, to be somewhat higher than is helpful for most new students to analysis,
and also just for mathematics.

We hope that this book can serve as a great analysis book, as Rudin has
written, but with a slightly lower level of struggle expected. We hope we can
provide more intuition and maybe some more figures than Rudin does.\todo{talk
more about who we expect to read this and what its main purpose is. a first
book for analysis? a book to read WITH Rudin?}

This book is for...\todo{who is this book for?}

The prereqs for this book are...\todo{what should we assume, and not assume,
the reader knows?}

\end{document}
