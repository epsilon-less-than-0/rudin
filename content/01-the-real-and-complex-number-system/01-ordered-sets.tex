\documentclass[../../templates/section]{subfiles}

\begin{document}

\section{Ordered Sets}\label{sec:ordered-sets}

Often when we talk about collections of things (people, cars, dogs, etc.) we
talk about how they compare to each other (height, top speed, cuteness
respectively). \emph{But}, we can only do this because we have a way in which
these objects relate to each other. My dog is \emph{of course} cuter than
yours, so I might say my dog is better than yours. Symbolically, I might write
this as
\begin{equation*}\label{eq:doges}
    \text{your dog} < \text{my dog}
\end{equation*}
where ``$<$'' can be read as ``is less cute than'' in this particular scenario,
but in others, it might mean ``has a lower top speed'' or really anything else
you can think of. We could even take all dogs and compare them in lots of
different ways, such as by weight, or tail length, or number of hairs, or...

There are a lot of ways, but the idea is that with a collection of objects we
often like to talk about how they relate to each other and how we can compare
objects of this underlying collection or set. The following definition puts
this in terms we will use through the rest of the book.

\begin{definition}\label{def:order}
    If we let $S$ be a set, then an \emph{order} on $S$ is a relation, often
    denoted $<$, with two extra properties.
    \begin{itemize}
        \item If $x$ and $y$ are in $S$, then \emph{only one} of the following
        is true.
        \[ x < y, \qquad x = y, \qquad y < x\]
        \item If $x$, $y$ and $z$ are in $S$ and $x < y$ and $y < z$, then $x <
        z$.
    \end{itemize}\todo{change bullets to match with rudin}
\end{definition}

Note that we did not define what the symbol $>$ means, but as is often done we
will use it because mathematics is nothing without some abuses of notation. If
we write $x > y$, take that to mean $y < x$, but instead you may read it as $x$
is ``greater than'' $y$ or $x$ is ``larger than'' $y$. Along with this
notation, we will use $x \le y$ to mean that $x$ is either less than $y$ or it
is equal to $y$, but we don't know which. Similarly with $\ge$.

While in English (and many other languages) we rely on context to understand
what set and order people are using when they talk, in mathematics we have to
be very pedantic. Hence the following definition.

\begin{definition}\label{def:ordered-set}
    An \emph{ordered set} is a set, together with an order defined on said set.
\end{definition}

Going back to the dogs, people often say ``Nate, you're dog is \emph{the
cutest}'' which would imply, if taken literally, that there is no dog that is
cuter than mine. People love these kind of extremes. We have a whole book
dedicated to people who are the \emph{most} at something (The Guinness World
Records) and it comes out every year. We also have the Olympics to find more of
the \emph{most} people. The fastest person on land, the fastest person in
water, the fastest person on land/water/wheel (triathlon). We love this kind of
thing, and of course some people are also interested in the slowest.

\end{document}
